%==============================================================================
% Voorbeeld gebruik documentklasse hogent-article
%==============================================================================
%
% Compileren in TeXstudio:
%
% - Zorg dat Biber de bibliografie compileert (en niet Biblatex)
%   Options > Configure > Build > Default Bibliography Tool: "txs:///biber"
% - F5 om te compileren en het resultaat te bekijken.
% - Als de bibliografie niet zichtbaar is, probeer dan F5 - F8 - F5
%   Met F8 compileer je de bibliografie apart.
%
% Als je JabRef gebruikt voor het bijhouden van de bibliografie, zorg dan
% dat je in ``biblatex''-modus opslaat: File > Switch to BibLaTeX mode.


\documentclass{hogent-article}
%\usepackage{graphicx}
%\graphicspath{ {./images/} }


% TODO: geef werktitel van je eigen voorstel op
\PaperTitle{Project management in systeembeheerprojecten}
% TODO: geef op welk soort artikel dit is
% Dit is typisch de opdracht en het vak waarvoor dit artikel geschreven is, bv.
% ``Verslag onderzoeksproject Onderzoekstechnieken 2018-2019''
\PaperType{Verslag onderzoeksproject Business Analysis 2021-2022}

% TODO: vul je eigen naam in als auteur, geef ook je emailadres mee!
\Authors{Nicholas De Wree\textsuperscript{1}, Ismail Karakaya\textsuperscript{2}, Nick Raff\textsuperscript{3}, Bram Stevens\textsuperscript{4}} % Authors

% TODO: vul de naam van je co-promotor in.
% Als het hier gaat om een voorstel voor de bachelorproef, dan ben je hier
% verplicht de naam van je co-promotor in te vullen. Zoniet, dan kan je het
% leeg laten.
\CoPromotor{}

% Contactinfo: Geef hier de contactgegevens van elke auteur van het artikel (en
% indien van toepassing ook van de co-promotor).
\affiliation{
    \textsuperscript{1} \href{mailto:nicholas.dewree@student.hogent.be}{nicholas.dewree@student.hogent.be}}
\affiliation{
    \textsuperscript{2} \href{mailto:ismail.karakaya@student.hogent.be}{ismail.karakaya@student.hogent.be}}
\affiliation{
    \textsuperscript{3}  \href{mailto:nick.raff@student.hogent.be }{nick.raff@student.hogent.be}}
\affiliation{
    \textsuperscript{4}
    \href{mailto:bram.stevens@student.hogent.be}{bram.stevens@student.hogent.be}}

%---------- Abstract ----------------------------------------------------------

\Abstract{De meeste project management methodologieën lijken opgezet met softwareontwikkeling of implementatie van pakketten in het achterhoofd. Zijn ze ook toepasbaar op systeembeheer, bijv. op een project voor de migratie naar een nieuwe versie van een OS? Wat zijn in de praktijk de gelijkenissen en de verschilpunten met projecten voor softwareontwikkeling of implementatie? Om meer onderzoek te doen naar dit onderwerp gaan wij enkele interviews uitvoeren bij mensen die meer kennis hebben over dit onderwerp. Ook gaan we ons zelf proberen verdiepen in dit onderwerp zodat we enkele gerichte vragen kunnen stellen aan deze mensen. Met deze interviews willen we ook graag te weten komen hoe we zelf als nieuwe werknemer het best kunnen omgaan in een project.
}

%---------- Onderzoeksdomein en sleutelwoorden --------------------------------
% TODO: Vul de sleutelwoorden aan.


\Keywords{Project management, systeembeheer, methodologieën}
\newcommand{\keywordname}{Sleutelwoorden} % Defines the keywords heading name

%---------- Titel, inhoud -----------------------------------------------------

\begin{document}
    \setcounter{tocdepth}{1}
    \flushbottom % Makes all text pages the same height
    \maketitle % Print the title and abstract box
    \tableofcontents % Print the contents section
    \thispagestyle{empty} % Removes page numbering from the first page
    
    %------------------------------------------------------------------------------
    % Hoofdtekst
    %------------------------------------------------------------------------------
    
    \section{Inleiding}
    
   In het verleden gingen IT projecten vaak fout. Waarom deze fout liepen kon tal van redenen hebben: slechte planning, slechte communicatie, te weinig budget etc. Hier moest iets aan veranderen, door gebruik te maken van nieuwe methodologieën zoals agile, scrum en kanban. Het grootste doel van deze methodologieën is zorgen voor een betere planning en een duidelijker overzicht. Ook worden vandaag de projecten helemaal anders aangepakt dan vroeger. Hierin is een goede communicatie en duidelijke rollen binnen een team cruciaal. Zeker omdat ook nu met al deze nieuwe inzichten een project tot een goed einde brengen een moeilijke opdracht blijft. Binnen het systeembeheer kunnen er nog andere problemen opduiken aangezien deze methodologieën vooral gebruikt worden binnen softwareontwikkeling. Desondanks kunnen deze methodologieën wel handig zijn binnen systeembeheer. Hoe dit juist aangepakt wordt verder duidelijk gemaakt verder in dit verslag.
    
    \section{Overzicht literatuur}
    
    % Refereren naar de literatuur kan met:
    % \autocite{BIBTEXKEY} -> (Auteur, jaartal)
    % \textcite{BIBTEXKEY} -> Auteur (jaartal)
    % Voorbeeld van een referentie~\autocite{Moore2002}
    
    Projectmanagement is de toepassing van processen, methoden, vaardigheden, kennis en ervaring om specifieke projectdoelstellingen te bereiken volgens de projectaanvaardingscriteria binnen overeengekomen parameters. Projectmanagement heeft eindproducten die beperkt zijn tot een eindige tijdschaal en budget. Het bestaat uit een groot aantal componenten en verschillende taken die moeten uitgevoerd worden. Een algemeen projectmanagement kan in 5 verschillende delen opgedeeld worden: planning (planning), initiation (initiatie), execution (uitvoering), monitoring (monitoring) en closing (sluiting). Een projectmanager is degene die deze operatie bestuurt. Een projectmanager is verantwoordelijk voor het plannen en organiseren van het project en het succesvol opleveren aan een team dat effectief aan het project zal moeten werken. Zij moeten ervoor zorgen dat het project binnen een bepaalde tijdspanne en budget tot een goed einde kan gebracht worden dat voldoet aan de eisen van de klant. 
    
    Een mislukt projectmanagement is een ongewenste catastrofe die organisaties graag willen vermijden. Een van de belangrijkste redenen die dit kunnen veroorzaken is het gebrek aan een gekwalificeerde projectmanager. Er zijn tal van redenen waarom projecten in problemen kunnen terechtkomen: hoge verwachtingen, te onrealistisch, slechte communicatie, planningen gebaseerd op onvoldoende gegevens, ontbrekende punten, onvoldoende details, slechte schattingen, risico’s die niet geïdentificeerd, verondersteld of beheerd worden, … Een organisatie mag dit soort gevaren niet over het hoofd zien als ze een project tot een goed eind willen brengen. Niveau van kennis, vaardigheden, persoonlijke karakteristieken, ervaring moeten zeker en vast beschouwd worden bij het aanwerven van een projectmanager. Deze worden de individuele competenties genoemd. Hoe beter de individuele competenties van een projectmanager, hoe beter dat hij/zij in staat zal zijn om een project te beheren. Daarom moeten de individuele competenties van een projectmanager goed geanalyseerd worden. Hiervoor kunnen organisaties gebruik maken van competentiegericht systemen bij het selecteren van individuen. Uit een onderzoek (Russell, 2001) gericht op Fortune 50 bedrijven (ranglijst van wereldwijde bedrijven met beste vooruitzichten voor aanhoudende groei) blijkt dat managers die geselecteerd werden m.b.v. een competentiegericht proces elk een extra 3 miljoen dollar aan jaarlijkse winst genereerden in vergelijking met andere managers die geselecteerd werden m.b.v. een proces dat niet gericht was op de competenties. Een projectmanager moet in staat zijn om de rol van een leider te nemen. Een goede leider kan goed communiceren. Ze weten heel goed wanneer ze moeten spreken of luisteren. Ze zijn ook in staat om dagelijks te onderhandelen met klanten, managers, teamleden, personeel en over de scope, veranderingen, contracten, opdrachten. Ze moeten ook op de hoogte zijn van de technologische vooruitgang zowel binnen als buiten het bedrijf. Er moet verwacht worden dat een projectmanager gebruik kan maken van projectplanning tools, e-mail en kalender tools, virtual meeting tools, … Ze hebben ook de vaardigheid nodig om problemen op te lossen die zowel kunnen optreden in normale – als kritische omstandigheden. Naast individuele competenties moet er uiteraard ook aandacht besteed worden aan de soft-skills/persoonlijke attributen. Enkele belangrijke soft skills die van een projectmanager verwacht mogen worden zijn: eerlijkheid, ambitie, intelligentie en zelfvertrouwen.
    
    Veel methodologieën voor projectbeheer zijn perfect toepasbaar voor softwareontwikkeling, maar hetzelfde kan niet volledig worden gezegd voor systeembeheer. Ondanks het feit dat beide in dezelfde sector plaats nemen, zijn hun rollen en taken verschillend. Softwareontwikkelaars zijn verantwoordelijk voor het ontwerpen, creëren en coderen van programma’s waarmee men in het dagdagelijks leven mee geconfronteerd wordt. Deze programma’s kunnen variëren van een eenvoudig mobiel- of computerprogramma tot complexe systeemprogramma's die van cruciaal belang zijn voor de werking van een bedrijfsproductiesysteem. Naast het leveren van een programma, zijn zij ook op lange termijn verantwoordelijk voor het onderhouden en updaten van het programma. 
    
    Een systeembeheerder is verantwoordelijk voor het opzetten en onderhouden van een systeem of server. De systeembeheerder zorgt nog voor andere zaken waaronder het installeren van software, netwerkcommunicatie monitoren, het opzetten van beveiliging voor nieuwe gebruikers, het beveiligen van servers om onbevoegde toegang te vermijden en nog veel meer. Met de jaren heeft de systeembeheerder een nieuwe verantwoordelijkheid gekregen, cloud computing. Hierdoor zijn er meer taken om uitgevoerd te worden. Het opzetten van een cloud-infrastructuur en netwerkdiensten in de cloud zijn voorbeelden van nieuwe taken. De meest essentiële skill van een systeembeheerder is problemen oplossen. Als een server of werkstation niet meer functioneert, wordt van hen verwacht dat zij het oplossen. Een systeembeheerder moet snel en correct een diagnose van het probleem opstellen. Natuurlijk moeten ze ook voorzien van back-up- en noodherstelplanningen om ervoor te zorgen dat de problemen zich zo weinig mogelijk voordoen. Het is belangrijk om te onthouden dat systeembeheerders geen ontwikkelaars of software engineers zijn. Zij moeten wel begrijpen hoe de software werkt.
       
    
    \section{Vragen voor het interview}    
    \subsection{Interview 1: Athos - Joeri Verhavert}
    De eerste persoon die we interviewen heet Joeri Verhavert, hij werkt momenteel bij Open-Future waar hij de functie van Linux SysAdmin beoefent. Bij Open-future doet hij vooral kleinere projecten. In het verleden werkte Joeri nog bij Athos waar hij ook Linux SysAdmin was. 
    \\
    \begin{enumerate}\bfseries
        \item Hoe ziet een typische werkdag eruit?
    \end{enumerate}

     We beginnen om 9 uur, daarna voeren we een paar taken uit. Om 9u30 is er een meeting waar we overleggen over issues, vorige issues, changes en planningen voor die dag. We geven ook opmerkingen aan elkaar, het is een soort stand up meeting. Na de meeting lossen we tickets op of werken we aan een project. Om 13u30 hebben we terug meeting zoals in de voormiddag maar korter. Daarna werken we terug verder aan tickets of aan een project.\\  
     
     
     \begin{enumerate}[resume]\bfseries
         \item Hoe verloopt de eerste meeting bij de aanvang van een project/opdracht? Worden de taken en het project goed gepresenteerd? Welke partijen zijn er bijtrokken? 
     \end{enumerate} 
 
      Er is de projectleider die alles in goede banen leidt. Enkele technische mensen die taken uitvoeren. Dit kan uit meerdere teams bestaan. Daarnaast is er ook een verantwoordelijke buiten de projectleider en soms zijn er een paar analisten. Bij aanvang wordt er besproken wat er moet gebeuren en bij de volgende meetings wordt de taakverdeling in orde gebracht.\\
    
    \begin{enumerate}[resume]\bfseries
        \item Hoe ziet jouw ideale team voor een project eruit? Welke persoonlijkheden heb je nodig en hoe kom je tot een ideale verstandhouding?  
    \end{enumerate}
    Mijn ideaal team zou uit technische mensen, een aantal testers en analisten bestaan. Je hebt ook zeker een sociale projectleider nodig die goed kan sturen en begeleiden. De persoonlijkheid van het team is niet zo belangrijk zolang ze weten waarover ze praten. Ik vind het belangrijk dat er communicatie en openheid is bij problemen zodat iedereen elkaar kan bijstaan.\\
    
    \begin{enumerate}[resume]\bfseries
        \item Kun je een voorbeeld geven waarbij een project fout is gelopen en wat was de reden hiervoor?  
    \end{enumerate}

    Er zijn nog geen projecten fout gelopen alhoewel ik wel vind dat een project beter kon. Hierbij was de tijdinschatting verkeerd ingeschat door de complexiteit van het project. We hadden hierdoor weinig tijd.\\
    
    \begin{enumerate}[resume]\bfseries
        \item Wat zijn de grootste obstakels die je kan tegenkomen binnen een project?  
    \end{enumerate}

    Tijd en zaken die op voorhand niet in orde geraakt zijn, bijvoorbeeld de aanpassingen van de servers of slechte configuraties.\\
    
    \begin{enumerate}[resume]\bfseries
        \item Hoe zorgt u ervoor dat een project blijft verlopen volgens planning? Welke technieken gebruikt u hiervoor? 
    \end{enumerate}

     Een techniek die we gebruiken is duidelijk inplannen in de agenda met bijhorende tijdsinschattingen per deeltaak. Timestamps gebruiken werkt hier goed bij.\\
     
     \begin{enumerate}[resume]\bfseries
         \item Welke projectmanagement methodologieën (Agile, Scrum, Waterfall, ...) worden toegepast binnen jullie team/organisatie? 
     \end{enumerate}
 
    Het gebruik van Jira maar dit is niet voor mijn team, wij gebruiken 4me en ItsMe. Echt agile wordt er bij ons niet gewerkt buiten in sommige developer teams. Vier keer per jaar worden er omgevingen in productie gebracht en via productielijnen die worden opgedeeld. Zij hebben elk hun eigen manieren van uitwerken en plannen opstellen.\\
    
     \begin{enumerate}[resume]\bfseries
         \item De meeste projectmanagement methodologieën lijken opgezet met software-ontwikkeling of implementatie van pakketten in het achterhoofd. Zijn ze ook toepasbaar op systeembeheer? 
     \end{enumerate}
    
    Er zijn veel gelijkenissen, je kan ze toepassen maar dat wordt niet vaak gedaan op dezelfde wijze, mits een lichte aanpassing is dat nog wel half bruikbaar. Waterfall lijkt de meest relevante methodologie, want bijvoorbeeld infrastructuur plaatsen kan je niet doen als je niet de juiste kabels hebt.\\
    
    \begin{enumerate}[resume]\bfseries
        \item Welke doelen of ervaringen zou u graag nog willen bereiken in de toekomst?
    \end{enumerate}

    Ik zou meer technologieën willen leren kennen, deelnemen in grotere projecten die in verschillende omgevingen zijn en meer certificaten behalen.\\
    
    \begin{enumerate}[resume]\bfseries
        \item Wat zouden tips zijn die u geeft aan een jonge medewerker die begint aan uw job? Wat zijn de belangrijkste kwaliteiten die deze persoon zeker onder de knie moet hebben.
    \end{enumerate}

    Leergierigheid vind ik een belangrijke eigenschap en je moet ook wat feeling hebben voor technologieën.

    \subsection{Interview 2: Signpost - Jyngvar Samyn}
    Onze tweede geïnterviewde heet Yngvar Samyn, hij werkt momenteel bij Signpost waar hij systeem- en netwerkbeheerder is maar ook nog andere taken vervult omdat ze te weinig werknemers hebben. In het verleden was hij consultant bij Tobania en service desk engineer (first line en second line) bij Unifiedpost en Nomadesk.\\☻
    
    \begin{enumerate}\bfseries
        \item Hoe ziet een typische werkdag eruit?
    \end{enumerate}

    We beginnen met dailies, back-ups checken en de checks van monitoring navolgen. Daarna een ticketting systeem, dit zijn de reports en klachten. Dan werken we aan grote projecten van infrastructuur. s’Avonds is er meestal permanentie en om de maand is er maintenance.\\
    
    \begin{enumerate}[resume]\bfseries
        \item Hoe verloopt de eerste meeting bij de aanvang van een project/opdracht? Worden de taken en het project goed gepresenteerd? Welke partijen zijn er bijtrokken? 
    \end{enumerate} 
    
    Er is een dagelijkse stand up waarin er wordt verteld wat er gebeurt. Door weinig interesse van hoger op doen mijn collega en ik het met ons twee en bespreken we de kosten in samenwerking met hardware, shared services en digitale methoden. Daarna tekenen we een schema uit, een stappenplan in Visio, tekenen we de topologie uit, documenteren we alles. Als dat gedaan is beginnen we aan de uitvoering.\\
    
    \begin{enumerate}[resume]\bfseries
        \item Hoe ziet jouw ideale team voor een project eruit? Welke persoonlijkheden heb je nodig en hoe kom je tot een ideale verstandhouding?  
    \end{enumerate}

    Ik werk eigenlijk nooit in een team maar goede afspraken en communicatie met een collega is key. Er moet een duidelijke transparantie over het werk zijn en men moet kunnen documenteren zodat beiden duidelijk weten wie waar staat.
    
    \begin{enumerate}[resume]\bfseries
        \item Kun je een voorbeeld geven waarbij een project fout is gelopen en wat was de reden hiervoor?  
    \end{enumerate}
    
    De fibers waren stuk gegaan door een niet IT-er, de fibers zaten vast in een muur via een kliksysteem over 25m. Na het vervangen struikelt hij over de kabel en maakt die stuk. Een ander geval was een collega die certificaten moest updaten op Linux en had per ongelijk ergens een \$ gezet waardoor alles down was. Dit had voorkomen kunnen worden omdat de devs niet gebruik maakten van een testomgeving.\\
    
    \begin{enumerate}[resume]\bfseries
        \item Wat zijn de grootste obstakels die je kan tegenkomen binnen een project?  
    \end{enumerate}
    
    Het afwijken van de planning of geannuleerd worden midden in het project. Vooral slechte planning en communicatie.\\
    
    \begin{enumerate}[resume]\bfseries
        \item Hoe zorgt u ervoor dat een project blijft verlopen volgens planning? Welke technieken gebruikt u hiervoor? 
    \end{enumerate}
    
    Gefocust blijven werken maar dat is moeilijk in de positie. Een goeie tip is afspraken maken en tijd vrijhouden. Heel veel communicatie en stand ups zijn ook belangrijk.\\
    
    \begin{enumerate}[resume]\bfseries
        \item Welke projectmanagement methodologieën (Agile, Scrum, Waterfall, ...) worden toegepast binnen jullie team/organisatie? 
    \end{enumerate}
    
    We hebben een canvasbord zoals Jira, maar we gebruiken Planner van Office365, het is een beetje zoals Trello.\\
    
    \begin{enumerate}[resume]\bfseries
        \item De meeste projectmanagement methodologieën lijken opgezet met software-ontwikkeling of implementatie van pakketten in het achterhoofd. Zijn ze ook toepasbaar op systeembeheer? 
    \end{enumerate}
    
    Niet echt omdat je eerder een gewone planner gebruikt in plaat van methodologieën.\\
    
    \begin{enumerate}[resume]\bfseries
        \item Welke doelen of ervaringen zou u graag nog willen bereiken in de toekomst?
    \end{enumerate}
    
    In de nabije toekomst zou ik graag meer certificaten halen en in de verre toekomst meer Cloud Computing en Azure ervaring. Databases ook maar dit is low priority. In het algemeen ook up to date blijven met de IT-wereld.\\
    
    \begin{enumerate}[resume]\bfseries
        \item Wat zouden tips zijn die u geeft aan een jonge medewerker die begint aan uw job? Wat zijn de belangrijkste kwaliteiten die deze persoon zeker onder de knie moet hebben.
    \end{enumerate}
    
    Kennis in verband met collega's en groepsdynamiek, een sterke communicatie en openheid naar collega's, altijd doorzetten. Tijdig pauzeren.
    
    \section{Analyse van het interview}
    
    Lorem ipsum dolor sit amet, consectetur adipiscing elit, sed do eiusmod tempor incididunt ut labore et dolore magna aliqua. Ut enim ad minim veniam, quis nostrud exercitation ullamco laboris nisi ut aliquip ex ea commodo consequat. Duis aute irure dolor in reprehenderit in voluptate velit esse cillum dolore eu fugiat nulla pariatur. Excepteur sint occaecat cupidatat non proident, sunt in culpa qui officia deserunt mollit anim id est laborum.
    
    
    \section{Reflectie}
    
    Lorem ipsum dolor sit amet, consectetur adipiscing elit, sed do eiusmod tempor incididunt ut labore et dolore magna aliqua. Ut enim ad minim veniam, quis nostrud exercitation ullamco laboris nisi ut aliquip ex ea commodo consequat. Duis aute irure dolor in reprehenderit in voluptate velit esse cillum dolore eu fugiat nulla pariatur. Excepteur sint occaecat cupidatat non proident, sunt in culpa qui officia deserunt mollit anim id est laborum.
    

%------------------------------------------------------------------------------
% Referentielijst
%------------------------------------------------------------------------------
% TODO: de gerefereerde werken moeten in BibTeX-bestand ``bibliografie.bib''
% voorkomen. Gebruik JabRef om je bibliografie bij te houden en vergeet niet
% om compatibiliteit met Biber/BibLaTeX aan te zetten (File > Switch to
% BibLaTeX mode)

\phantomsection
\printbibliography[heading=bibintoc]
\end{document}
