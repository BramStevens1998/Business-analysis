%==============================================================================
% Voorbeeld gebruik documentklasse hogent-article
%==============================================================================
%
% Compileren in TeXstudio:
%
% - Zorg dat Biber de bibliografie compileert (en niet Biblatex)
%   Options > Configure > Build > Default Bibliography Tool: "txs:///biber"
% - F5 om te compileren en het resultaat te bekijken.
% - Als de bibliografie niet zichtbaar is, probeer dan F5 - F8 - F5
%   Met F8 compileer je de bibliografie apart.
%
% Als je JabRef gebruikt voor het bijhouden van de bibliografie, zorg dan
% dat je in ``biblatex''-modus opslaat: File > Switch to BibLaTeX mode.

\documentclass{hogent-article}

\usepackage{lipsum} % Voor vultekst

%------------------------------------------------------------------------------
% Metadata over het artikel
%------------------------------------------------------------------------------

%---------- Titel & auteur ----------------------------------------------------

% TODO: geef werktitel van je eigen voorstel op
\PaperTitle{project management in systeembeheerprojecten}
% TODO: geef op welk soort artikel dit is
% Dit is typisch de opdracht en het vak waarvoor dit artikel geschreven is, bv.
% ``Verslag onderzoeksproject Onderzoekstechnieken 2018-2019''
\PaperType{Verslag onderzoeksproject Business Analysis 2021-2022}

% TODO: vul je eigen naam in als auteur, geef ook je emailadres mee!
\Authors{Stevens Bram\textsuperscript{4}} % Authors

% TODO: vul de naam van je co-promotor in.
% Als het hier gaat om een voorstel voor de bachelorproef, dan ben je hier
% verplicht de naam van je co-promotor in te vullen. Zoniet, dan kan je het
% leeg laten.
\CoPromotor{}

% Contactinfo: Geef hier de contactgegevens van elke auteur van het artikel (en
% indien van toepassing ook van de co-promotor).
\affiliation{
  \textsuperscript{1} \href{}{}}
\affiliation{
  \textsuperscript{2} \href{}{}}
\affiliation{
    \textsuperscript{3} \href{}{}}
\affiliation{
    \textsuperscript{4} \href{mailto:bram.stevens@student.hogent.be}{bram.stevens@student.hogent.be}}

%---------- Abstract ----------------------------------------------------------

\Abstract{

}

%---------- Titel, inhoud -----------------------------------------------------

\begin{document}

\flushbottom % Makes all text pages the same height
\maketitle % Print the title and abstract box
\tableofcontents % Print the contents section
\thispagestyle{empty} % Removes page numbering from the first page

%------------------------------------------------------------------------------
% Hoofdtekst
%------------------------------------------------------------------------------

\section{Inleiding}


\section{Overzicht literatuur}

% Refereren naar de literatuur kan met:
% \autocite{BIBTEXKEY} -> (Auteur, jaartal)
% \textcite{BIBTEXKEY} -> Auteur (jaartal)



\section{Methodologie}



\section{Experimenten}


\section{Analyse resultaten}



\section{Conclusie}



%------------------------------------------------------------------------------
% Referentielijst
%------------------------------------------------------------------------------
% TODO: de gerefereerde werken moeten in BibTeX-bestand ``bibliografie.bib''
% voorkomen. Gebruik JabRef om je bibliografie bij te houden en vergeet niet
% om compatibiliteit met Biber/BibLaTeX aan te zetten (File > Switch to
% BibLaTeX mode)

\phantomsection
\printbibliography[heading=bibintoc]

\end{document}
